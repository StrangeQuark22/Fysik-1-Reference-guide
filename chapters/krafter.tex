\section{Tyngdkraften}
Tyngdkraften, bättre känd som gravitationen, är den kraft som två massor utgör på varandra enligt fysikens lagar. Den betecknas $F_g$ och har samma enhet som alla andra krafter. Formeln för tyngdkraften är följande:
\begin{equation*}
    F_g = G \cdot \frac{m_1m_2}{r^2}
\end{equation*}
$G$ är \emph{gravitationskonstanten} och har ett värde på $G \approx 6.67 \cdot 10^{-11}$, $m_1$ och $m_2$ är massorna av de två interagerande föremålen och $r$ är avstånden mellan deras masscentrum. Detta samband innebär att $F_g \hyperref[def:propto]{\propto} \frac{1}{r^2}$. Detta gör att tyngdkraft avtar myhcket snabbt när avståndet ökar.

\section{Friktionskraft}
Friktionskraften är den kraft som två föremål utgör på varandra när de är i kontakt och en yttre kraft verkar på ena föremålet medan en lika stor kraft inte verkar på det andra. Denna har beteckningen $F_f$ och har formeln
\begin{equation*}
    F_f = \mu N
\end{equation*}